% THIS IS SIGPROC-SP.TEX - VERSION 3.1
% WORKS WITH V3.2SP OF ACM_PROC_ARTICLE-SP.CLS
% APRIL 2009
%
% It is an example file showing how to use the 'acm_proc_article-sp.cls' V3.2SP
% LaTeX2e document class file for Conference Proceedings submissions.
% ----------------------------------------------------------------------------------------------------------------
% This .tex file (and associated .cls V3.2SP) *DOES NOT* produce:
%       1) The Permission Statement
%       2) The Conference (location) Info information
%       3) The Copyright Line with ACM data
%       4) Page numbering
% ---------------------------------------------------------------------------------------------------------------
% It is an example which *does* use the .bib file (from which the .bbl file
% is produced).
% REMEMBER HOWEVER: After having produced the .bbl file,
% and prior to final submission,
% you need to 'insert'  your .bbl file into your source .tex file so as to provide
% ONE 'self-contained' source file.
%
% Questions regarding SIGS should be sent to
% Adrienne Griscti ---> griscti@acm.org
%
% Questions/suggestions regarding the guidelines, .tex and .cls files, etc. to
% Gerald Murray ---> murray@hq.acm.org
%
% For tracking purposes - this is V3.1SP - APRIL 2009

\documentclass{acm_proc_article-sp}

\makeatletter
\let\@copyrightspace\relax
\makeatother

\usepackage{tabularx}

\begin{document}

\title{Accountability and Privacy in the Modern Internet}
\subtitle{Comsys Seminar Paper WS 2015/16}

\numberofauthors{1}
\author{
\alignauthor
Lucas Braun\\
	RWTH Aachen University \\
       \email{lucas.braun@rwth-aachen.de}
}

\maketitle
\begin{abstract}
%Your typical abstract.
\end{abstract}

\section{Introduction}
The history of the modern Internet as we know it today goes back as far as the early eighties -- IPv4, ICMP and TCP, which where all specified in 1981, together with 1984's DNS, still are the foundation on which all Internet traffic is being transported.

The way these technologies are designed is heavily reflecting the challenges that were of importance at that time -- namely "the creation of a distributed communication network that is robust against packet loss and other network failures; support across multiple types of networks and communication services; and the management of Internet resources in a cost-effective and distributed way" \cite{mot}.

As time went by and the Internet began to grow bigger and bigger, new challenges that where not incorporated into the original design arose. The following work will focus on two of the biggest of those: accountability and privacy.

While there is already a plethora of technologies, that enable for varying levels of privacy, e.g., Tor, NAT and the likes, accountability is on a completely different page. Not only is accountability hard to guarantee, but it tends to be mutually exclusive with privacy. This is largely due to the fact that the first usually means strengthening sources addresses while the latter usually means weakening them \cite{apip}.

To delve deeper into these topics, section~\ref{sec:acc} explains what accountability is, why it is needed and by whom and present some recent research projects while section~\ref{sec:priv} does the same for privacy. Furthermore, section~\ref{sec:apip} presents the \emph{Accountable and Private Internet Protocol}, a recent attempt at creating a solution that balances between accountabilty and privacy \cite{apip}.

[...]

%Explain what accountability and privacy actually are in the context of internet, motivate why anyone would need it (maybe case examples), who would need it and why no one has it.

\section{Accountability}
\cite{bootstrapping} \cite{mirkovic}
\label{sec:acc}
A lot of crime and chaos in the real world is being prevented because we have mechanisms in place that associate distinct persons or group of persons with actions, making everyone accountable for most of what they are doing. The Internet, as we have it today, is largely devoid of such mechanisms. Techniques like source IP spoofing can give everyone easy anonymity and enables for malicious actions such as denial of service attacks without having to fear any consequences.

Naylor et al. \cite{apip} define three key properties that must be fulfilled in an accountable internet:
\begin{itemize}
\item For every packet there exists an entity that takes responsibility if the packet is of malicious nature
\item If a malicious flow is detected, it can be stopped quickly
\item Subsequent mischief can be prevented by punishment or expulsion
\end{itemize}

Accountability would thus be a welcomed trait in the Internet, as it can greatly help to stop and even prevent attacks. What is needed to provide such accountability is some efficient way to verify the alleged identity of a source, something that is not done directly on the network layer of today's internet. 

Approaches to fix these shortcomings exist, but many of them come with their own shortcomings, rendering them not ideal for real world deployment, e.g., complicated mechanisms changing "the free-access model of the Internet", the need for external sources of trust, or the need for manual filtering by network operators \cite{aip}.

%There are a lot of cool papers. AIP must be explained, maybe IPA or Mirkovic and Reiher as a second approach (or even both?).
\subsection{Accountable Internet Protocol}
\label{sec:aip}
Andersen et al. propose AIP \cite{aip}, the \emph{Accountable Internet Protocol}, as a replacement for the current IP. Hosts and domains using AIP are enabled to prove their identity without the need for an external trusted authority.

\subsubsection{Basic design}
\label{sec:aipbd}

In AIP, the transition to prefixed CIDR-style addresses is reverted in favor of a simpler multi component address. Each independently administered network gets divided into one or more \emph{accountability domains} (ADs) by its administrative unit, providing each AD with a globally unique identifier. Likewise, every end host must be provided a globally unique EID identifier, leading to an address scheme of \texttt{AD:EID}. In case of need for hierarchically organized accountability domains, the scheme can be extended to \texttt{AD$_1$:AD$_2$:\ldots:AD$_k$:EID}. Furthermore, if a host connects to an accountability domain via multiple network interfaces, each interface is provided its own EID by altering the \emph{interface bits} -- the last eight bits of the EID.

\begin{figure}[h!]
	\label{fig:aipadr}
	\newcolumntype{C}{>{\centering}X}
	\begin{tabularx}{0.47\textwidth}{|c|C|c|}
		\hline & & \\
 		Crypto vers & Public key hash & Interface \\
		8 bits & 144 bits & 8 bits \\ & & \\
		\hline
	\end{tabularx}
	\caption{AIP addresses consist of a version number, a hash of a public key and an interface identifier 		(which is set to zero for accountability domains) \cite{aip}}
\end{figure}

This simplification of the addressing enables for \emph{self-certifying} addresses: the main part of the 160 bit long AIP address is simply a 144 bit long hash of the endpoint's respective accountability domain's public key (Figure \ref{fig:aipadr}).

\subsubsection{Verification}

Now, when an end point tries to send a packet, it is the first-hop router's duty to verify the end point's identity. When a packet from a not-verified host arrives, the packet is dropped and the router replies with a \emph{verification packet} V. This packet is signed with a periodically created router specific secret. The sending host signs this packet as well, returns it to the router, which can now verify the sender's identity with the help of his private key, permitting the sender to resend the dropped packet as well as subsequent packets.


\section{Privacy}
\label{sec:priv}
\cite{blind} \cite{tor} \cite{lap}
\subsection{Tor instead of IP}
\subsection{end-to-end encryption}
\subsection{NAT}

\section{Accountable and Private Internet Protocol}
\label{sec:apip}

With the \emph{Accountable and Private Internet Protocol} (APIP) \cite{apip}, as the name suggests, Naylor et al. try to find a way to balance between the seemingly mutually exclusive properties accountability and privacy. To achieve this goal they borrow heavily from AIP (see section \ref{sec:aip}), but delegate the job of verifying identities to so called \emph{accountability delegates}. The way this is done opens up the possibility to mask ones return address and thus allows for the application of privacy preserving methods like end-to-end encryption.

The basic idea that enables APIP to have that functionality is to separate the accountability address from the return address. It is now no longer necessary to know who sent a packet to be able to verify that it is accounted for, since this is done exclusively with the accountability delegate.

\subsection{Addressing and packet flow}

In APIP, addresses are always of the form \texttt{NID:HID:SID}. The three components of such an address are as follows:
\begin{itemize}
\item \emph{network id} \texttt{NID}: used to identify the correct domain
\item \emph{host id} \texttt{HID}: used to find the host inside of it's domain. The \texttt{HID} is assumed to be self-certifying, as explained in section \ref{sec:aipbd}
\item \emph{socket id} \texttt{SID}: to further specify the host's socket used for the packet
\end{itemize}
Notice that this doesn't make any assumptions about the underlying protocol, but rather is a generic addressing scheme which is applicable to most protocols. For example, when applied to IP, \texttt{NID} and \texttt{HID} are represented by an IP address while \texttt{SID} corresponds to the port number. However, the assumption about the \texttt{HID} being self-certifying must be relaxed to accommodate APIP with IP, which in turn leads to the need for a Public-Key-Infrastructure. 

Depending on it's type, each packet uses either two or three APIP addresses. The two mandatory ones are the \emph{destination address} and the \emph{accountability address}, which is pointing to the accountability delegate. The optional third address is the \emph{return address}, which is used if a response to the packet is expected and is not needed otherwise.

An implication of this approach is that flow IDs might not be fine granular enough if a single accountability delegate vouches for several clients, but this can be fixed by either using distinct SIDs for every client and/or by adding a flow ID field to the packet header. How this is handled does have an impact on both privacy and flow control; while just allocating the same flow ID to every client might provide maximum anonymity, it can also lead to a lot of non-malicious traffic being blocked when only a single client misbehaves. Then again, assigning a unique flow ID to every client might provide very fine flow control but it also makes it easy to map traffic to specific senders. A good trade off between these extremes is posed by assigning each client not a single unique flow ID but a pool of enough flow IDs to ensure a sufficient level of anonymity. 

\subsection{Accountability mechanism}

The verification process is somewhat similar to AIP, but additionally uses a level of indirection: When sending a package, the sender also informs it's accountability delegate about said package by calling the delegate's \texttt{brief(packet, clientID)} interface. Any on-path router as well as the packet's receiver can act as a verifier by calling the delegate's \texttt{verify(packet)} interface. Verifiers are than free to filter traffic that can not be verified and furthermore receivers can use the delegate's \texttt{shutoff(packet)} interface to permanently keep the delegate from vouching for a specific flow.

\subsubsection{Brief()}
Informing the accountability delegate about new packets is done by either briefing each packet individually with the transmission of the packet's fingerprint or by periodically briefing them in bulk with the transmission of a bloom filter of several fingerprints. The fingerprints themselves are of the following form:
\begin{equation}
F(P) = H(K_{SD_S} || P_{header} || H(P_{body} ))
\end{equation}
where $H$ is a cryptographic hash function and $K_{SD_S}$ is a symmetric key only known to the sender and the delegate.

To circumvent the need for verification on the briefs, a token field is included in the briefs' header. This could for example be a hash chain of a shared secret between sender and delegate. The delegate will know that is has to look for this token, since briefs' accountability and destination addresses are identical.

\paragraph{Recursive Verification}
Instead of using the aforementioned briefing mechanism, it is also possible to do recursive verification and merely use the accountability delegate as a middleman. In this case, \texttt{brief()} is not used. Instead, the delegate forwards all calls to \texttt{verify()} to the packet's alleged sender who in turn has to decide whether he indeed sent the packet or not and respond accordingly to the delegate. While this approach saves network and storage overhead it also has a drawback; in order for this to work, every flow ID must be mapped to exactly one client, potentially reducing client anonymity.

\subsubsection{Verify()}
When a packet passes a router, it checks whether the packet's flow ID has already been verified during the current \emph{verification interval}. At the end of each interval all verifications get purged. If it has not yet been verified, the router will verify the packet with the accountability delegate specified in the header and may choose to drop the packet and inform the sender about the ongoing verification process. The accountability delegate than checks whether the packet in question has been briefed by it's sender, the used \texttt{SID} is associated with the sender the connection between sender and receiver has not been blacklisted with a call to \texttt{shutoff()}. In case that all three conditions are met, the verifier adds the flow to it's whitelist for the current verification interval.

\subsubsection{Shutoff()}
As previously mentioned, 

\subsection{Privacy mechanism}

\subsection{Evaluation}

\subsubsection{Performance}

\section{Summary}

\section{Conclusion}

%
% The following two commands are all you need in the
% initial runs of your .tex file to
% produce the bibliography for the citations in your paper.
\bibliographystyle{abbrv}
\bibliography{apip}  % sigproc.bib is the name of the Bibliography in this case
% You must have a proper ".bib" file
%  and remember to run:
% latex bibtex latex latex
% to resolve all references
%
% ACM needs 'a single self-contained file'!
%
%APPENDICES are optional
%\balancecolumns
\balancecolumns
% That's all folks!
\end{document}
