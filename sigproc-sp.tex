% THIS IS SIGPROC-SP.TEX - VERSION 3.1
% WORKS WITH V3.2SP OF ACM_PROC_ARTICLE-SP.CLS
% APRIL 2009
%
% It is an example file showing how to use the 'acm_proc_article-sp.cls' V3.2SP
% LaTeX2e document class file for Conference Proceedings submissions.
% ----------------------------------------------------------------------------------------------------------------
% This .tex file (and associated .cls V3.2SP) *DOES NOT* produce:
%       1) The Permission Statement
%       2) The Conference (location) Info information
%       3) The Copyright Line with ACM data
%       4) Page numbering
% ---------------------------------------------------------------------------------------------------------------
% It is an example which *does* use the .bib file (from which the .bbl file
% is produced).
% REMEMBER HOWEVER: After having produced the .bbl file,
% and prior to final submission,
% you need to 'insert'  your .bbl file into your source .tex file so as to provide
% ONE 'self-contained' source file.
%
% Questions regarding SIGS should be sent to
% Adrienne Griscti ---> griscti@acm.org
%
% Questions/suggestions regarding the guidelines, .tex and .cls files, etc. to
% Gerald Murray ---> murray@hq.acm.org
%
% For tracking purposes - this is V3.1SP - APRIL 2009

\documentclass{acm_proc_article-sp}

\makeatletter
\let\@copyrightspace\relax
\makeatother

\begin{document}

\title{Accountability and Privacy in the Modern Internet}
\subtitle{Comsys Seminar Paper WS 2015/16}

\numberofauthors{1}
\author{
\alignauthor
Lucas Braun\\
	RWTH Aachen University \\
       \email{lucas.braun@rwth-aachen.de}
}

\maketitle
\begin{abstract}
Your typical abstract.
\end{abstract}

\section{Introduction}
The history of the modern Internet as we know it today goes back as far as the early eighties -- IPv4, ICMP and TCP, which where all specified in 1981, together with 1984's DNS, still are the foundation on which all Internet traffic is being transported.

The way these technologies are designed is heavily reflecting the challenges that were of importance at that time -- namely "the creation of a distributed communication network that is robust against packet loss and other network failures; support across multiple types of networks and communication services; and the management of Internet resources in a cost-effective and distributed way" [CITEME MOTI].

As time went by and the Internet began to grow bigger and bigger, new challenges that where not incorporated into the original design arose. The following work will focus on two of the biggest of those: accountability and privacy.

While there are already a plethora of technologies that enable for varying levels of privacy, like Tor, NAT and the likes, accountability is on a completely different page. Not only is accountability hard to guarantee, but it tends to be mutually exclusive with privacy. This is largely due to the fact that the first usually means strengthening sources addresses while the latter usually means weakening them [CITEME APIP].

%Explain what accountability and privacy actually are in the context of internet, motivate why anyone would need it (maybe case examples), who would need it and why no one has it.

\section{Goals}

\section{Accountability}
There are a lot of cool papers. AIP must be explained, maybe IPA or Mirkovic and Reiher as a second approach (or even both?).
\subsection{Accountable Internet Protocol}

\section{Privacy}
\subsection{Tor instead of IP}
\subsection{end-to-end encryption}
\subsection{NAT}

\section{Accountable and Private Internet Protocol}

\section{Summary}

\section{Conclusion}

%
% The following two commands are all you need in the
% initial runs of your .tex file to
% produce the bibliography for the citations in your paper.
\bibliographystyle{abbrv}
\bibliography{sigproc}  % sigproc.bib is the name of the Bibliography in this case
% You must have a proper ".bib" file
%  and remember to run:
% latex bibtex latex latex
% to resolve all references
%
% ACM needs 'a single self-contained file'!
%
%APPENDICES are optional
%\balancecolumns
\balancecolumns
% That's all folks!
\end{document}
